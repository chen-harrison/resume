\documentclass{resume}

\begin{document}

\name{Harrison Chen}

\info{email}{hchen.robotics@gmail.com}
\info{phone}{+1(734)968-5295}
\info{location}{New York, NY}
% \urlinfo{website}{chen-harrison.github.io}{https://chen-harrison.github.io}
\urlinfo{linkedin}{chen-harrison}{https://linkedin.com/in/chen-harrison}
\urlinfo{github}{chen-harrison}{https://github.com/chen-harrison}

\section{Experience}

\entry{Autonomy Engineer}{PDW}{New Rochelle, NY}{Mar 2022 – Present}
\begin{details}
    \item Augmented quadcopter’s autonomous navigation stack with trajectory generation capable of obstacle avoidance and breadth-first search for safe start and goal positions, increasing safety and reliability while flying
    \item Modularized navigation pipeline to accommodate multiple flight modes with unique implementations, improving code abstraction and ease of adding future algorithms
    \item Integrated multi-sensor OctoMap and Voxblox 3D mapping algorithms into mapping ROS package, allowing drone to map 3D occupancy by combining input from forward- and downward-facing RealSense cameras
    \item Evaluated stereo depth DNNs on Arducam sensors in search of learning-based vision alternatives using Nvidia TAO Toolkit and MATLAB Deep Learning Toolbox for model modification and tuning
\end{details}

\entry{Robotics Engineer}{Jugaad Labs}{Philadelphia, PA}{Mar 2021 – Mar 2022}
\begin{details}
    \item Contributed to development of automotive situational awareness system for semi-trucks, using Python and ROS to identify and monitor nearby vehicles using center point-based object detection and Kalman filter tracking
    \item Built application with Nvidia Isaac SDK to perform object detection on camera feeds in Isaac Sim warehouse environment, serving as a theoretical sensing foundation for autonomous logistics
\end{details}

\entry{Applied Product Development Intern}{FANUC America}{Rochester Hills, MI}{Jun 2020 – Aug 2020}
\begin{details}
    \item Strengthened functionality for ArcTool recovery mechanism using proprietary programming language Karel, enabling welding robots to recalibrate in any reachable end effector pose
\end{details}

\entry{Student / Team Member}{Robotic Systems Laboratory Course (ROB 550)}{Ann Arbor, MI}{Jan 2020 – May 2020}
\begin{details}
    \item Implemented a simulated SLAM robot in C++ with 2D occupancy grid mapping, odometry motion model, beam measurement model, Monte Carlo localization, and A* path planning
    \item Collaborated with teammates on the development of an inverted pendulum robot using C and RCL, including PID control for balancing, manual steering via joystick, and autonomous movement along series of waypoints
\end{details}

\section{Education}

\entry{Master of Science in Robotics}{University of Michigan}{Ann Arbor, MI}{Dec 2020}
\begin{details}
    \item \textbf{GPA: 3.96/4.00}
    \item Relevant coursework: Mobile Robotics, Deep Learning for Computer Vision, Robot Modeling and Control
\end{details}

\entry{Bachelor of Science in Mechanical Engineering}{Northwestern University}{Evanston, IL}{June 2019}
\begin{details}
    \item \textbf{GPA: 3.80/4.00}
    \item Relevant coursework: Advanced Mechatronics, Feedback Systems
    \item Activities: Education Chair @ Refresh Dance Crew, Social Chair @ Chinese Students Association, Tau Beta Pi
\end{details}

\section{Skills \& Interests}
\textbf{Programming:} C++, Python, MATLAB, Bash, Git \\
\textbf{Robotics:} ROS, Eigen, OpenCV, PCL, PyTorch, 3D geometry, kinematics, Bayesian statistics, sensor calibration \\
\textbf{Interests:} soccer, running, dance, cooking, environmental conservation

\end{document}